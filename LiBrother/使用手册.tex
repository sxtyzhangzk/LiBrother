\documentclass[UFT8]{ctexart}
\usepackage{CJK}
\begin{document}
\title{$LiBrother$使用手册}
\author{张哲恺\quad 冼藏越洋\quad 徐遥\quad 王辰\quad 张嘉恒\quad 朱奕\\2015级ACM班\\上海交通大学}
\maketitle
\tableofcontents
\section{简介}
$LiBrother$系统操作手册,其主要的作用在于为用户提供系统的使用方法和技巧,帮助用户更好更快的了解系统,使用系统,以及解答用户的一些使用问题。
\section{使用时的流程图}
\includegraphics[width=0.9\linewidth]{8}
\section{所包括的主要页面}
$LiBrother$的主要页面包括主页面,管理员页面和用户页面
\section{学会使用主页面}
\includegraphics[width=0.9\linewidth]{1.png}\\
在主页面中,可以支持用书名,ID,作者等关键字查询图书馆图书。在滑块中用户选择利用图书的某一特性来查询,在下面边框中输入信息,点击查询,之后就会出现该书目的详细信息。同时主页面也支持已有用户的登陆和新用户的注册。用户可以根据不同需要进行选择。其他所有的页面都要通过主页面进入。
\section{用户注册界面}
\includegraphics[width=0.9\linewidth]{2.png}\\
在此页面下,用户只需要输入个人基本信息就可以注册成功,如果有信息错误,就会弹出新的警告框指出用户错误。
\section{用户主页面}
\includegraphics[width=0.9\linewidth]{3.png}\\
当用户登录后,会进入用户主页面,不同等级的用户可以根据页面所提供的选择,结合自己的目的,点击按钮,进入新的页面或者进行查询。
\section{用户借书的操作页面}
\includegraphics[width=0.9\linewidth]{4}\\
用户打开选择某本书后,可以进一步获取该书目基本信息和简单介绍,之后选择借阅或者放弃,还书页面与其相似。
\section{修改密码的例子}
\includegraphics[width=0.9\linewidth]{5}\\
用户想要修改密码时,可以在用户主页面选择修改密码按钮,进入修改密码的界面,经过新密码确认后,用户的新密码就设置成功了。如果有错误就会弹出窗口提示用户错误信息。
\section{管理员页面}
\includegraphics[width=0.9\linewidth]{7}\\
如果通过用户登录,系统根据用户名发现用户事图书管理员,那么用户就会进入管理员页面,然后点击相应的按钮进行操作。但是普通用户是没有权限进入管理员页面的。
\section{管理员设置图书信息}
\includegraphics[width=0.9\linewidth]{6}\\
当管理员要修改图书信息时,点击修改图书信息按钮就会进入上述页面,然后就图书的相关信息进行设置,就完成了新图书的插入。如果有部分信息没有填写或者出现错误,系统会弹出警告框提醒管理员。
\section{总结}
用户在使用$LiBrother$时,一定要根据页面提供的按钮进行操作,一旦出现错误,注意看弹出的错误信息,根据提示进行修改,有任何问题请发邮件到$729267402@qq.com$

本使用手册最终解释权归二哥的图书馆的设计者所有。
\end{document}

